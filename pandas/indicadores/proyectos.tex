\documentclass{article}
\usepackage[utf8]{inputenc}
\pagestyle{empty}
\begin{document}
\begin{itemize}
\item Masas y mezclas de neutrinos en modelos 3-N-1 (N=3,4)
\begin{itemize}
\item Investigador principal: William Ponce
\item Fecha de inicio y tipo: 2014, Convocatoria Ciencias Exactas
\end{itemize}
\item Implicaciones de modelos tipo seesaw radiativo en las fronteras de la física de partículas y la cosmología
\begin{itemize}
\item Investigador principal: Óscar Alberto Zapata Noreña
\item Fecha de inicio y tipo: 2014, Convocatoria Ciencias Exactas
\end{itemize}
\item Estrategia para la Sostenibilidad GFIF 2013-2014
\begin{itemize}
\item Investigador principal: Diego Restrepo
\item Fecha de inicio y tipo: 2013, Sostenibilidad
\end{itemize}
\item Implicaciones de modelos supersimétricos con violación de paridad R en las fronteras de física de partículas y cosmología
\begin{itemize}
\item Investigador principal: Diego Restrepo
\item Fecha de inicio y tipo: 2013, Colciencias
\end{itemize}
\item Materia oscura de gravitinos y masas de neutrinos en modelos con violación trilineal y bilineal inducida de paridad R
\begin{itemize}
\item Investigador principal: Óscar Alberto Zapata Noreña
\item Fecha de inicio y tipo: 2012, Profesores Recién Vinculados
\end{itemize}
\item Testing the Standar Cosmological Model Studyng the effects of not gaussianity and large scale inhomogenities
\begin{itemize}
\item Investigador principal: Antonio Enea Romano
\item Fecha de inicio y tipo: 2012, Profesores Recién Vinculados
\end{itemize}
\item Fenomenología de modelos supersimétricos con ruptura de paridad R y violación de número leptónico.
\begin{itemize}
\item Investigador principal: Diego Alejandro Restrepo Quintero
\item Fecha de inicio y tipo: 2012, Mediana Cuantía
\end{itemize}
\end{itemize}
\end{document}
